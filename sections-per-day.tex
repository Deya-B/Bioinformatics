% Created 2024-09-17 Tue 14:34
% Intended LaTeX compiler: pdflatex
\documentclass[12pt]{article}
\usepackage[utf8]{inputenc}
%\usepackage[T1]{fontenc}
\usepackage{graphicx}
\usepackage{longtable}
\usepackage{wrapfig}
\usepackage{rotating}
\usepackage[normalem]{ulem}
\usepackage{amsmath}
\usepackage{amssymb}
\usepackage{capt-of}
\usepackage{hyperref}
%\pagestyle{empty}
\usepackage[margin=1.9cm]{geometry}
\usepackage{gitinfo2}
\renewcommand{\familydefault}{\sfdefault}
\setcounter{secnumdepth}{0}
\author{Ramon Diaz-Uriarte}
\date{\today}
\title{Statistics part: what we will cover each day}
\date{\gitAuthorDate\ {\footnotesize (Release\gitRels: Rev: \gitAbbrevHash)}}
\hypersetup{
 pdfauthor={Ramon Diaz-Uriarte},
 pdftitle={},
 pdfkeywords={},
 pdfsubject={},
 pdfcreator={Emacs 30.0.60 (Org mode 9.7.5)},
 pdflang={English}}
\begin{document}

\maketitle
\section*{How to read this}
``Day 2'' refers to the second session or lesson of the classes devoted to stats. For example, that will most likely be 13-November-2024. (This notation allows us to reuse this file, even if we change the exact date of a lesson).


``Section 5'': refers to the section in the main PDF, ``R-basic-stats.pdf'', titled ``Some basic statistics with R''. To be redundant, I often give in parentheses a short version of the section (e.g., ``section 5 (plots)'', where ``plots'' stands for ``Looking at the data: plots'' that is the full, long, section title of section number 5).

\subsection*{Speed}
\label{sec:speed}

The plan below is not the ideal plan. The ideal plan includes the ``omics'' slides and the categorical data analysis PDF, as well as a programming example of writing the code for a permutation test, which would be included at the end. And we want to go over what is an R package too. Thus, I will try to go slightly faster than what is shown below. This will be possible if you read the notes before coming to class.


\section*{Day 1}
\label{sec:org21b08e0}
\begin{itemize}
\item Sections: 5 (plots), 6 (two-sample t; including supplementary PDF about confidence intervals referred to in section 6.2), 7 (one and two-tailed), 8 (power)

  \item For those interested, we will discuss \textbf{at the end of class (i.e., expect to stay after 19:00 on 11-November-2024)}, any doubts about the projects for the practical programming exercise.
\end{itemize}
\section*{Day 2}
\label{sec:org5a4c730}
\begin{itemize}
\item Whatever remains from Day 1. Sections: 9 (equiv. testing), 10 (bayesian), 11 (conf. int: see longer slides).
\end{itemize}
\section*{Day 3}
\label{sec:org609aabb}
\begin{itemize}
\item Paired stuff: 12.1, 12.2.1, 12.2.2, 12.2.3, 12.2.8, 12.3, 12.4, 12.5 (plots for paired), 12.6, 13 (one-sample), 14 (non-par), 15 (non-indep data), 16 (symmetry and paired t). 12.7 (a first taste of lin.mods. ---only if time). Start section 18. \textbf{Read the external files on your own BEFORE class, I will answer questions about them, but I won't go over them in class.}.
\end{itemize}
\section*{Day 4 }
\label{sec:org394fbf9}
\begin{itemize}
\item Whatever remains from Day 3: Section 18, including the two external files (anova basic theory and anova theory even simpler); section 19 (FWER and FDR), some 20 (two-way anova). \textbf{Look at the notes for section 20 BEFORE class}.

   Unless you look at the notes \textbf{before you come to class} this will not make any sense. The two-way anova section is long (more than 50 pages), possibly completely new, and possibly not intuitive nor easy.

\end{itemize}
\section*{Day 5 }
\label{sec:orgcb0a12f}
\begin{itemize}
\item sections: 20 (two-way anova), 21 (regression).


  Do not expect to understand this starting from 0 in just two hours of class. Again, come to class having looked at the material, \textbf{even if during the on your own, before-class, reading it seems confusing}.
\end{itemize}
\section*{Day 6 }
\label{sec:org6f322e8}
\begin{itemize}
\item Sections 22 (multiple regression), 23 (ancova), 24 (interactions, summary)
\end{itemize}
\section*{Day 7 }
\label{sec:org2142285}
\begin{itemize}
\item Sections 25 (diagnostics), 26 (variable/model selection), 28 (experimental design)
\end{itemize}
\section*{Day 8 }
\label{sec:org3d49cf0}
\begin{itemize}
\item Section 29 (causal inference: see additional PDF --- \textbf{really read the additional PDF on your own before coming to class}; otherwise, this will make no sense in just the time we have in class).
\item If time left,  R packages and miscell programming stuff (e.g., permutation test); 30 (ratios).
\end{itemize}
\section*{Day 9: the hour after the statistics exam}
\begin{itemize}
\item R packages and miscell programming stuff; omics slides; categorical data analysis; 30 (ratios). (Obviously, we cannot do all of this in one hour). \textbf{The exam will last one hour, and we will use the remining hour to go over these topics.}
\end{itemize}
\label{sec:org32ab79d}

\end{document}
%%% Local Variables:
%%% mode: latex
%%% TeX-master: t
%%% End:
